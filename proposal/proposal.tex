\documentclass[twoside]{scrartcl}
\usepackage[utf8]{inputenc}
\usepackage[T1]{fontenc}
\usepackage{lmodern}
\usepackage{latexsym}
\usepackage{amsfonts}
\usepackage{amssymb}
\usepackage{fancyhdr,lastpage}
\usepackage{hyperref}
\usepackage{amsmath}
\usepackage{geometry}
\usepackage[ngerman]{babel}
\geometry{hmargin={2cm,2cm},vmargin={2.4cm,3cm}}
\usepackage{comment}
\newcommand{\llfloor}{\left\lfloor}
\newcommand{\rrfloor}{\right\rfloor}
\setlength{\headheight}{20pt}
\pagestyle{fancy}
\fancyhf{}
\fancyhead[L]{Algorithm Proposal}
\fancyhead[C]{Fabian Hirschmann, Michael Markert}
\fancyhead[R]{\today}
\fancyfoot[C]{Page \thepage\ of \pageref{LastPage}}
\begin{document}
\section{Definition: Qualität}
Die Qualität eines Algorithmus zur Streckengeneralisierung
wird durch den Abstand von der ursprünglichen Streckenführung
beschrieben.\\\\
Mit zunehmender Reduzierung der Stützpunkte einer Streckenführung
nimmt die Qualität der generalisierten Streckenführung für
gewöhnlich ab.
\section{Vorauswahl}
Bahnhöfe werden in einen Status versetzt, in dem sie von keinen
Algorithmus aus der Streckenführung entfernt werden können.
\section{Algorithmen}
\subsection{Algorithmus 1: Combination}
Die Stützpunkte zweier Streckenführungen
$s_1, s_2 \in \{(x_0, y_0), (x_1, y_1), \ldots, (x_n, y_n)\}$
werden angeglichen indem die Stützpunkte
$a \in s_1$ und  $b \in s_2$ $a = b$ gesetzt wird, falls
$|a - b| < \varepsilon$ gilt.\\
\\
Dies dient hauptsächlich dazu, dass nahe beieinander liegende
Streckenverläufe nach Anwendungen weiterer Algorithmen auch
wieder beieinander liegen.
\subsection{Algorithmus 2: Ramer-Douglas-Peucker Algorithmus}
Durch die Anwendung des Ramer-Douglas-Pleucker Algorithmus
wird durch Reduzierung der Stützpunkte eine Vereinfachung
der Streckenführung erreicht.

\subsection{Algorithmus 3}
Dieser Algorithmus besteht aus der sequentiellen Anwendung aller
vorhergehenden Algorithmen.

\end{document}
